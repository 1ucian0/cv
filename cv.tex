%%%%%%%%%%%%%%%%%%%%%%%%%%%%%%%%%%%%%%%%%%%%%%%%%%%%%%%%%%%%%%%%%%%%%%%%
%%%%%%%%%%%%%%%%%%%%%% Simple LaTeX CV Template %%%%%%%%%%%%%%%%%%%%%%%%
%%%%%%%%%%%%%%%%%%%%%%%%%%%%%%%%%%%%%%%%%%%%%%%%%%%%%%%%%%%%%%%%%%%%%%%%

%%%%%%%%%%%%%%%%%%%%%%%%%%%%%%%%%%%%%%%%%%%%%%%%%%%%%%%%%%%%%%%%%%%%%%%%
%% NOTE: If you find that it says                                     %%
%%                                                                    %%
%%                           1 of ??                                  %%
%%                                                                    %%
%% at the bottom of your first page, this means that the AUX file     %%
%% was not available when you ran LaTeX on this source. Simply RERUN  %% 
%% LaTeX to get the ``??'' replaced with the number of the last page  %% 
%% of the document. The AUX file will be generated on the first run   %%
%% of LaTeX and used on the second run to fill in all of the          %%
%% references.                                                        %%
%%%%%%%%%%%%%%%%%%%%%%%%%%%%%%%%%%%%%%%%%%%%%%%%%%%%%%%%%%%%%%%%%%%%%%%%

%%%%%%%%%%%%%%%%%%%%%%%%%%%% Document Setup %%%%%%%%%%%%%%%%%%%%%%%%%%%%

% Don't like 10pt? Try 11pt or 12pt
\documentclass[10pt]{article}

% This is a helpful package that puts math inside length specifications
\usepackage{calc}

% Layout: Puts the section titles on left side of page
\reversemarginpar

%
%         PAPER SIZE, PAGE NUMBER, AND DOCUMENT LAYOUT NOTES:
%
% The next \usepackage line changes the layout for CV style section
% headings as marginal notes. It also sets up the paper size as either
% letter or A4. By default, letter was used. If A4 paper is desired,
% comment out the letterpaper lines and uncomment the a4paper lines.
%
% As you can see, the margin widths and section title widths can be
% easily adjusted.
%
% ALSO: Notice that the includefoot option can be commented OUT in order
% to put the PAGE NUMBER *IN* the bottom margin. This will make the
% effective text area larger.
%
% IF YOU WISH TO REMOVE THE ``of LASTPAGE'' next to each page number,
% see the note about the +LP and -LP lines below. Comment out the +LP
% and uncomment the -LP.
%
% IF YOU WISH TO REMOVE PAGE NUMBERS, be sure that the includefoot line
% is uncommented and ALSO uncomment the \pagestyle{empty} a few lines
% below.
%

%% Use these lines for letter-sized paper
% \usepackage[paper=letterpaper,
%             %includefoot, % Uncomment to put page number above margin
%             marginparwidth=1.2in,     % Length of section titles
%             marginparsep=.05in,       % Space between titles and text
%             margin=1in,               % 1 inch margins
%             includemp]{geometry}

%% Use these lines for A4-sized paper
\usepackage[paper=a4paper,
           %includefoot, % Uncomment to put page number above margin
           marginparwidth=30.5mm,    % Length of section titles
           marginparsep=1.5mm,       % Space between titles and text
           margin=25mm,              % 25mm margins
           includemp]{geometry}

%% More layout: Get rid of indenting throughout entire document
\setlength{\parindent}{0in}

%% This gives us fun enumeration environments. compactenum will be nice.
\usepackage{paralist}

\usepackage[T1]{fontenc}
\usepackage[medfamily,opticals,mathlf,minionint,loosequotes,mathtabular]{MinionPro}
\usepackage[scaled=.9]{blgothic}
\usepackage{microtype}


%% Reference the last page in the page number
%
% NOTE: comment the +LP line and uncomment the -LP line to have page
%       numbers without the ``of ##'' last page reference)
%
% NOTE: uncomment the \pagestyle{empty} line to get rid of all page
%       numbers (make sure includefoot is commented out above)
%
\usepackage{fancyhdr,lastpage}
\pagestyle{fancy}
%\pagestyle{empty}      % Uncomment this to get rid of page numbers
\fancyhf{}\renewcommand{\headrulewidth}{0pt}
\fancyfootoffset{\marginparsep+\marginparwidth}
\newlength{\footpageshift}
\setlength{\footpageshift}
          {0.5\textwidth+0.5\marginparsep+0.5\marginparwidth-2in}
\lfoot{\hspace{\footpageshift}%
       \parbox{4in}{\, \hfill %
                    \arabic{page} of \protect\pageref*{LastPage} % +LP
%                    \arabic{page}                               % -LP
                    \hfill \,}}

% Finally, give us PDF bookmarks
\usepackage{color,hyperref}
\definecolor{darkblue}{rgb}{0.0,0.0,0.3}
\hypersetup{colorlinks,breaklinks,
            linkcolor=darkblue,urlcolor=darkblue,
            anchorcolor=darkblue,citecolor=darkblue}

%%%%%%%%%%%%%%%%%%%%%%%% End Document Setup %%%%%%%%%%%%%%%%%%%%%%%%%%%%


%%%%%%%%%%%%%%%%%%%%%%%%%%% Helper Commands %%%%%%%%%%%%%%%%%%%%%%%%%%%%

% The title (name) with a horizontal rule under it
%
% Usage: \makeheading{name}
%
% Place at top of document. It should be the first thing.
\newcommand{\makeheading}[1]%
        {\hspace*{-\marginparsep minus \marginparwidth}%
         \begin{minipage}[t]{\textwidth+\marginparwidth+\marginparsep}%
                {\large \bfseries #1}\\[-0.15\baselineskip]%
                 \rule{\columnwidth}{.5pt}%
         \end{minipage}}

% The section headings
%
% Usage: \section{section name}
%
% Follow this section IMMEDIATELY with the first line of the section
% text. Do not put whitespace in between. That is, do this:
%
%       \section{My Information}
%       Here is my information.
%
% and NOT this:
%
%       \section{My Information}
%
%       Here is my information.
%
% Otherwise the top of the section header will not line up with the top
% of the section. Of course, using a single comment character (%) on
% empty lines allows for the function of the first example with the
% readability of the second example.
\renewcommand{\section}[2]%
        {\pagebreak[2]\vspace{1.3\baselineskip}%
         \phantomsection\addcontentsline{toc}{section}{#1}%
         \hspace{0in}%
         \marginpar{
         \raggedright \scshape #1}#2}

% An itemize-style list with lots of space between items
\newenvironment{outerlist}[1][\enskip\textbullet]%
        {\begin{enumerate}[#1]}{\end{enumerate}%
         \vspace{-.6\baselineskip}}

% An itemize-style list with little space between items
\newenvironment{innerlist}[1][\enskip\textbullet]%
        {\begin{compactenum}[#1]}{\end{compactenum}}

% To add some paragraph space between lines.
% This also tells LaTeX to preferably break a page on one of these gaps
% if there is a needed pagebreak nearby.
\newcommand{\blankline}{\quad\pagebreak[2]}

%%%%%%%%%%%%%%%%%%%%%%%% End Helper Commands %%%%%%%%%%%%%%%%%%%%%%%%%%%

%%%%%%%%%%%%%%%%%%%%%%%%% Begin CV Document %%%%%%%%%%%%%%%%%%%%%%%%%%%%

\begin{document}
\makeheading{{\LARGE Arnar~Birgisson}
  \hspace{.2cm}
  {\LARGE\sc curriculum~vitae}
  \hspace{\fill}
  {\normalsize\textmd{updated \today}}}

\section{contact information}
%
% NOTE: Mind where the & separators and \\ breaks are in the following
%       table.
%
% ALSO: \rcollength is the width of the right column of the table 
%       (adjust it to your liking; default is 1.85in).
%
\newlength{\rcollength}\setlength{\rcollength}{1.85in}%
%
\begin{tabular}[t]{@{}p{\textwidth-\rcollength}p{\rcollength}}
\href{http://www.chalmers.se/}%
     {Department of Computer Science and Engineering} 
                           & \textit{Phone:} (+46) 31 772 1062 \\
Chalmers University of Technology
                           & \textit{Mobile:} (+46) 72 040 1784 \\
R�nnv�gen 6B            & \textit{E-mail:}
\href{mailto:arnarb@chalmers.se}{arnarb@chalmers.se},\\
41296 G�teborg           & \hspace{29pt}
\href{mailto:arnarbi@gmail.com}{arnarbi@gmail.com}\\
SWEDEN & \hspace{29pt} \href{http://www.hvergi.net/arnar/}{www.hvergi.net/arnar}\\
\end{tabular}


\section{citizenship}
%
Icelandic

% \section{Research Interests}
% %
% Control theory, communication theory, behavioral ecology, cooperation
% theory, engineering education 

\section{education}
%
\href{http://www.chalmers.se/}{\textbf{Chalmers University of Technology}}, 
G�teborg, Sweden
\begin{outerlist}

\item[] Ph.D., Computer Science, ongoing since August 2009
        \begin{innerlist}
        \item Supervisor: 
            \href{http://www.math.chalmers.se/~andrei/}
                   {Andrei Sabelfeld}
        \item Areas of Study: Language based security, type based information flow analysis,
            application of information flow analysis to web security
        \end{innerlist}
\end{outerlist}

\blankline

\href{http://www.ru.is/}{\textbf{Reykjavik University}}, 
Reykjavik, Iceland
\begin{outerlist}

\item[] M.Sc., Computer Science, June 2009
        \begin{innerlist}
        \item Thesis Topic: \href{http://www.hvergi.net/arnar/projects/msc-thesis}{Topics in Structural Operational Semantics}
        \item Supervisor: 
              \href{http://www.ru.is/faculty/luca/}
                   {Luca Aceto}
        \item Areas of Study: Process Algebra, Operational Semantics
        \end{innerlist}
\end{outerlist}

\blankline

\href{http://www.hi.is/}{\textbf{University of Iceland}}, 
Reykjavik, Iceland
\begin{outerlist}

\item[] B.Sc., Mathematics, June 2004
        \begin{innerlist}
        \item Emphasis on Computer Science
        \end{innerlist}

\end{outerlist}

% \section{Awards and honours} 
% %
% \href{http://www.nsf.gov/}{National Science Foundation}
% \begin{innerlist}
% \item \href{http://www.nsfgk12.org/}{GK-12 Fellowship}, 2006
% \item \href{http://www.nsf.gov/grfp}
%            {Graduate Research Fellowship} Honorable Mention, 2005
% \end{innerlist}

% \blankline

% \href{http://www.osu.edu}{The Ohio State University}
% \begin{innerlist}
% \item \href{http://www.gradsch.osu.edu/Content.aspx?Content=44&itemid=2}
%            {Dean's Distinguished University Fellowship}, 2004
% \item Electrical and Computer Engineering Bradshaw Scholarship,
%         2002--2004
% \item Electrical and Computer Engineering Shafstall Scholarship,
%         2001--2003
% \item University Scholarship, 1999--2003
% \end{innerlist}

\section{academic experience}
\href{http://www.chalmers.se}{\textbf{Chalmers University of Technology}}, 
G�teborg, Sweden
\begin{outerlist}
\item[] \textit{Graduate Student}%
    \hfill \textbf{August 2009 to present}
\begin{innerlist}
    \item Participated in the Marktoberdorf Summer School of
    Logics and Languages for Reliability and Security, Germany, Agust 2009.
    \item Participated in the FOSAD Summer School in Bertinoro, Italy, September 2009.
    \end{innerlist}
\item[] \textit{Teaching Assistant}%
            \hfill \textbf{November 2009 to present}
\begin{innerlist}
    \item {\em Data Structures}, administered weekly exercise lessons, graded
        and supervised lab work.
    \item {\em Programming Languages}, graded and supervised lab work.
    \end{innerlist}
\end{outerlist}

\blankline

\href{http://www.ru.is}{\textbf{Reykjavik University}}, 
Reykjavik, Iceland
\begin{outerlist}
\item[] \textit{Graduate Student}%
        \hfill \textbf{September 2007 to June 2009}
\begin{innerlist}
\item Research assistantship from January 2008
\item Visiting Technical University of Eindhoven (TU/e) for two months
    during fall semester of 2008 for research related to M.Sc. thesis.
\end{innerlist}

\item[] \textit{Teaching Assistant}%
        \hfill \textbf{January 2008 to April 2009}
\begin{innerlist}
\item {\em Algorithm Design and Analysis}, weekly grading of assignments
  and preparation and presentation of solutions in class.
\item {\em Computer Science for Engineers II}, administering daily lab sessions
  in a three-week intensive course on programming, assisted with creation
  of excercises and exams.
\item {\em Introduction to Artificial Intelligence}, weekly lab sessions,
  creation of exercises, assistance with programming projects and grading.
\end{innerlist}
\end{outerlist}

\blankline

\href{http://www.hi.is}{\textbf{University of Iceland}}, 
Reykjavik, Iceland
\begin{outerlist}
\item[] \textit{Undergraduate Student}%
        \hfill \textbf{September 2001 to June 2004}
\end{outerlist}

\section{publications}
%
\textbf{Conference and workshop papers}
\begin{outerlist}
\item Arnar Birgisson, Mohan Dhawan, �lfar Erlingsson,
Vinod Ganapathy, Liviu Iftode.
{\em Enforcing Authorization Policies using Transactional Memory Inspection}.
ACM Conference on Computer and Communications Security (CCS 2008),~pp.~223--234.
2008.
%
\item Luca Aceto, Arnar Birgisson, Anna Ingolfsdottir, MohammadReza Mousavi and Michel Reniers.
{\em Rule Formats for Determinism and Idempotency.}  
LNCS proceedings of Fundamentals of Software Engineering 2009.
%
\item Arnar Birgisson and �lfar Erlingsson.
{\em An Implementation and Semantics for Transactional Memory Introspection in Haskell.}
ACM SIGPLAN Fourth Workshop on Programming Languages and Analysis for Security (PLAS 2009).
\end{outerlist}

\blankline

\textbf{Technical reports}
\begin{outerlist}
\item Arnar Birgisson, Mohan Dhawan, �lfar Erlingsson,
Vinod Ganapathy, Liviu Iftode.
{\em Enforcing Authorization Policies using Transactional Memory Inspection}.
Technical report, Rutgers University, Dept. of C.S. USA. DCS-TR-$628$.
{\em (Superseded by CCS 2008 conference paper.)}
%
\item Arnar Birgisson, �lfar Erlingsson.
{\em An Implementation and Semantics for Transactional Memory Introspection in Haskell}.
Technical report, Reykjavik University, School of Computer Science, Iceland. RUTR-CS$08007$.

\end{outerlist}

\section{professional experience}
%
\href{http://www.kvos.is/}{\textbf{Kvos}} (Oddi Printing Ltd. before
company restructure in 2006), 
Reykjavik, Iceland
\begin{outerlist}

\item[] \textit{IT Systems Analyst}%
        \hfill \textbf{June 2004 to December 2007}
\begin{innerlist}
\item Designed and implemented specialised in-house web-based IT systems
  for billing and production planning, human resources among others.
\item Implemented and maintained IT infrastructure for web-based
  information systems.
\item Design and maintenance on several mission-critical databases.
\item Partly responsible for design and maintenance of networking
  and telephone infrastructure.
\item Technical consultant on automated layout and desktop publishing.
  Helped with various jobs dealing with computer-driven layout.
\end{innerlist}

\item[] \textit{Part-time IT systems designer and programmer}%
        \hfill \textbf{May 2000 to June 2004}
\begin{innerlist}
\item Designed and implemented web-based information systems for
  production planning and management, billing, website content
  management.
\item Implemented and maintained Linux based infrastructure for 
  several web-based information systems.
\end{innerlist}
\end{outerlist}

\blankline

\href{http://www.visir.is/}{\textbf{visir.is}}, Reykjavik, Iceland
\begin{outerlist}
\item[] \textit{Web application programmer}%
        \hfill \textbf{July 1998 to April 2000}
\begin{innerlist}
\item Custom web development for high-volume Icelandic news website.
    Included database design and maintenance.
\end{innerlist}
\end{outerlist}

\blankline

\textbf{MotorIs}, 
Reykjavik, Iceland
\begin{outerlist}

\item[] \textit{Production assistant}%
        \hfill \textbf{summers of 1996--1998}
\begin{innerlist}
\item Various jobs in producing a weekly 30 minute show on Icelandic
  motor sport aired internationally. Jobs included cameraman, non-linear
  editing, sound mixing and graphical design.
\item Assisting in administration of rally and off-road tournaments,
  including implementing simple systems for score keeping, public
  announcements etc.
\end{innerlist}
\end{outerlist}

\blankline

\textbf{Various}, Eskifj�r�ur, Iceland \hfill \textbf{summers before 1996}
\begin{outerlist}
\item[] Various summer jobs as a teenager in fish factories, public service and commerce.
\end{outerlist}



% \section{technical skills} 
% %
% Programming (advanced): Python, C, C++, Java, SQL. Good generic
% and cross-language programming experience, ability to work with
% diverse environments.

% \blankline

% Programming (intermediate): Pascal, Perl, PHP, Scheme, Haskell, PostScript,
% UNIX shell scripting and others. 

% \blankline

% Operating Systems: Microsoft Windows XP/2000, Apple OS X, Linux,
%         HP Unix, and other UNIX variants

\section{personal research statement}
%
Already, many critical parts of modern society rest on top of several layers
of computer software and hardware, and things will only become more complex
in the future. As with other engineering efforts, I believe that such systems
cannot meet the requirements of reliability and maintainability unless they are
built on solid, well specified and sound foundations.

To address this, and preferably further the state of the art, my research interests
lie in the various topics surrounding programming languages and language based
security. Programming languages
provide system designers with the necessary high-level abstractions to realize their
designs, and as such, programming languages and the related technologies need to be
all things to everyone. They must be expressive enough to allow the designer to
create a complete and maintainable system. They must be simple, so that the systems
they express can be understood and maintained. They must be built on solid foundations
which provide assurances of correctness, in many cases through mathematical verification.
All of this is important in the context of security, where language technology can
significantly contribute to the task of designing and implementing safe systems.

% I believe at the moment, functional and declarative programming languages deserve to
% be continued to be studied in detail, as they fit these requirements better in many
% ways that imperative languages cannot. I also believe that the use of formal specifications
% of their semantics (to the extent possible) is beneficial, both to build verifiable
% systems as well as providing language designers important insight into their languages.

\blankline

The main purpose of programming languages is to provide a mapping from a high-level
specification of a system to, to a lower level. The lower level can be abstract,
mathematical models of execution - or it can be hardware or even a lower layer of
software (actually the most common case). I find great joy in studying such mappings,
designing new ones and finding ways to improve them. 
A particularly useful application of such research is in security. Security issues
cross-cut the abstraction levels of systems, i.e. a system that appears safe on one
abstraction level may very well not be on another one. By building in helpful
features in the mapping between layers, namely programming languages, it is possible
to make it significantly easier to write verifiably safe systems.
% This particular joy sparks
% my interest in a wide range of topics, from low level compilation and interpretation
% of languages, to high level and abstract specifications of language semantics.

My experience and interest in the practicalities of programming and computer
architecture fuses rather well with my fascination for the theoretical aspect.
By combining these two, I'd like to gain even deeper understanding of the current
state of the art to be able to find ways to advance it.

\blankline

% To be more specific, my current research interest lies across process algebra,
% structural operational semantics and functional programming (preferably lazy,
% pure and strongly typed, e.g. Haskell). I am interested in using these areas
% of study to work on language based security, modelling and verification and efficient
% and correct execution (dynamic or compiled).

% \blankline

% For a further description of my recent and current projects, as well as my
% qualifications for job offerings, please refer to the cover letter included
% in my application.

\section{references}
    For letters of references, please contact any of the following persons.
    \begin{itemize}
        \item Andrei Sabelfeld \\
            Associate Professor, Chalmers University of Technology \\
            {\tt http://www.math.chalmers.se/\string~andrei/} \\
            {\tt andrei@chalmers.se}
        \item Luca Aceto \\
            Professor, Reykjavik University \\
            {\tt http://www.ru.is/faculty/luca/} \\
            {\tt luca@ru.is}
        \item �lfar Erlingsson \\
            Associate Professor, Reykjavik University and Researcher, Microsoft Research \\
            {\tt http://www.ru.is/faculty/ulfar/} \\
            {\tt ulfar@ru.is}
        \item MohammadReza Mousavi\\
            Assistant Professor, Eindhoven University of Technology \\
            {\tt http://www.win.tue.nl/\string~mousavi/} \\
            {\tt m.r.mousavi@tue.nl}
    \end{itemize}
\end{document}

%%%%%%%%%%%%%%%%%%%%%%%%%% End CV Document %%%%%%%%%%%%%%%%%%%%%%%%%%%%%
